\documentclass[]{article}
\usepackage{lmodern}
\usepackage{amssymb,amsmath}
\usepackage{ifxetex,ifluatex}
\usepackage{fixltx2e} % provides \textsubscript
\ifnum 0\ifxetex 1\fi\ifluatex 1\fi=0 % if pdftex
  \usepackage[T1]{fontenc}
  \usepackage[utf8]{inputenc}
\else % if luatex or xelatex
  \ifxetex
    \usepackage{mathspec}
  \else
    \usepackage{fontspec}
  \fi
  \defaultfontfeatures{Ligatures=TeX,Scale=MatchLowercase}
\fi
% use upquote if available, for straight quotes in verbatim environments
\IfFileExists{upquote.sty}{\usepackage{upquote}}{}
% use microtype if available
\IfFileExists{microtype.sty}{%
\usepackage{microtype}
\UseMicrotypeSet[protrusion]{basicmath} % disable protrusion for tt fonts
}{}
\usepackage{hyperref}
\hypersetup{unicode=true,
            pdftitle={Data format for punching alien stowaway data},
            pdfauthor={Jens Åström},
            pdfborder={0 0 0},
            breaklinks=true}
\urlstyle{same}  % don't use monospace font for urls
\usepackage{color}
\usepackage{fancyvrb}
\newcommand{\VerbBar}{|}
\newcommand{\VERB}{\Verb[commandchars=\\\{\}]}
\DefineVerbatimEnvironment{Highlighting}{Verbatim}{commandchars=\\\{\}}
% Add ',fontsize=\small' for more characters per line
\usepackage{framed}
\definecolor{shadecolor}{RGB}{248,248,248}
\newenvironment{Shaded}{\begin{snugshade}}{\end{snugshade}}
\newcommand{\AlertTok}[1]{\textcolor[rgb]{0.94,0.16,0.16}{#1}}
\newcommand{\AnnotationTok}[1]{\textcolor[rgb]{0.56,0.35,0.01}{\textbf{\textit{#1}}}}
\newcommand{\AttributeTok}[1]{\textcolor[rgb]{0.77,0.63,0.00}{#1}}
\newcommand{\BaseNTok}[1]{\textcolor[rgb]{0.00,0.00,0.81}{#1}}
\newcommand{\BuiltInTok}[1]{#1}
\newcommand{\CharTok}[1]{\textcolor[rgb]{0.31,0.60,0.02}{#1}}
\newcommand{\CommentTok}[1]{\textcolor[rgb]{0.56,0.35,0.01}{\textit{#1}}}
\newcommand{\CommentVarTok}[1]{\textcolor[rgb]{0.56,0.35,0.01}{\textbf{\textit{#1}}}}
\newcommand{\ConstantTok}[1]{\textcolor[rgb]{0.00,0.00,0.00}{#1}}
\newcommand{\ControlFlowTok}[1]{\textcolor[rgb]{0.13,0.29,0.53}{\textbf{#1}}}
\newcommand{\DataTypeTok}[1]{\textcolor[rgb]{0.13,0.29,0.53}{#1}}
\newcommand{\DecValTok}[1]{\textcolor[rgb]{0.00,0.00,0.81}{#1}}
\newcommand{\DocumentationTok}[1]{\textcolor[rgb]{0.56,0.35,0.01}{\textbf{\textit{#1}}}}
\newcommand{\ErrorTok}[1]{\textcolor[rgb]{0.64,0.00,0.00}{\textbf{#1}}}
\newcommand{\ExtensionTok}[1]{#1}
\newcommand{\FloatTok}[1]{\textcolor[rgb]{0.00,0.00,0.81}{#1}}
\newcommand{\FunctionTok}[1]{\textcolor[rgb]{0.00,0.00,0.00}{#1}}
\newcommand{\ImportTok}[1]{#1}
\newcommand{\InformationTok}[1]{\textcolor[rgb]{0.56,0.35,0.01}{\textbf{\textit{#1}}}}
\newcommand{\KeywordTok}[1]{\textcolor[rgb]{0.13,0.29,0.53}{\textbf{#1}}}
\newcommand{\NormalTok}[1]{#1}
\newcommand{\OperatorTok}[1]{\textcolor[rgb]{0.81,0.36,0.00}{\textbf{#1}}}
\newcommand{\OtherTok}[1]{\textcolor[rgb]{0.56,0.35,0.01}{#1}}
\newcommand{\PreprocessorTok}[1]{\textcolor[rgb]{0.56,0.35,0.01}{\textit{#1}}}
\newcommand{\RegionMarkerTok}[1]{#1}
\newcommand{\SpecialCharTok}[1]{\textcolor[rgb]{0.00,0.00,0.00}{#1}}
\newcommand{\SpecialStringTok}[1]{\textcolor[rgb]{0.31,0.60,0.02}{#1}}
\newcommand{\StringTok}[1]{\textcolor[rgb]{0.31,0.60,0.02}{#1}}
\newcommand{\VariableTok}[1]{\textcolor[rgb]{0.00,0.00,0.00}{#1}}
\newcommand{\VerbatimStringTok}[1]{\textcolor[rgb]{0.31,0.60,0.02}{#1}}
\newcommand{\WarningTok}[1]{\textcolor[rgb]{0.56,0.35,0.01}{\textbf{\textit{#1}}}}
\IfFileExists{parskip.sty}{%
\usepackage{parskip}
}{% else
\setlength{\parindent}{0pt}
\setlength{\parskip}{6pt plus 2pt minus 1pt}
}
\setlength{\emergencystretch}{3em}  % prevent overfull lines
\providecommand{\tightlist}{%
  \setlength{\itemsep}{0pt}\setlength{\parskip}{0pt}}
\setcounter{secnumdepth}{0}
% Redefines (sub)paragraphs to behave more like sections
\ifx\paragraph\undefined\else
\let\oldparagraph\paragraph
\renewcommand{\paragraph}[1]{\oldparagraph{#1}\mbox{}}
\fi
\ifx\subparagraph\undefined\else
\let\oldsubparagraph\subparagraph
\renewcommand{\subparagraph}[1]{\oldsubparagraph{#1}\mbox{}}
\fi

%%% Use protect on footnotes to avoid problems with footnotes in titles
\let\rmarkdownfootnote\footnote%
\def\footnote{\protect\rmarkdownfootnote}

%%% Change title format to be more compact
\usepackage{titling}

% Create subtitle command for use in maketitle
\newcommand{\subtitle}[1]{
  \posttitle{
    \begin{center}\large#1\end{center}
    }
}

\setlength{\droptitle}{-2em}
  \title{Data format for punching alien stowaway data}
  \pretitle{\vspace{\droptitle}\centering\huge}
  \posttitle{\par}
  \author{Jens Åström}
  \preauthor{\centering\large\emph}
  \postauthor{\par}
  \predate{\centering\large\emph}
  \postdate{\par}
  \date{18 September, 2020}


%\linespread{1.3}


% You know, for landscape
\usepackage{lscape}
\usepackage{pdfpages}


% pandoc does not parse latex env - https://groups.google.com/forum/?fromgroups=#!topic/pandoc-discuss/oZETB5Ii1Cw
\newcommand{\blandscape}{\begin{landscape}}
\newcommand{\elandscape}{\end{landscape}}

% Make new page before each section
%\let\stdsection\section
%\renewcommand\section{\newpage\stdsection}

% Highlight inline `code`
\usepackage{soul}
\usepackage{xcolor}

\definecolor{Light}{gray}{.97}
\sethlcolor{Light}

\let\OldTexttt\texttt
\renewcommand{\texttt}[1]{\OldTexttt{\hl{#1}}}


\clubpenalty=10000      %kara za sierotki
\widowpenalty=10000  % nie pozostawiaj wdów
\brokenpenalty=10000    % nie dziel wyrazów miêdzy stronami
\exhyphenpenalty=999999   % nie dziel s³ów z myœlnikiem
\righthyphenmin=3     % dziel minimum 3 litery

\renewcommand{\topfraction}{0.95}
\renewcommand{\bottomfraction}{0.95}
\renewcommand{\textfraction}{0.05}
\renewcommand{\floatpagefraction}{0.35}

\begin{document}
\maketitle

{
\setcounter{tocdepth}{2}
\tableofcontents
}
\hypertarget{intro}{%
\section{Intro}\label{intro}}

This script creates examples of the tables in the planteimport-database,
which can be used for punching new data. The idea is that importing the
data could could be faster if the data was punched in this format from
the start. Currently I have to do a lot of rearranging and changing of
names manually.

\hypertarget{container-table}{%
\section{Container table}\label{container-table}}

This table holds the information of the sampled containers. This is
already punched in a very similar format to the table in the database,
so we could just continue using this format. Whenever there's a new
locality or new exporter, we also need to import these into their
respective lookup tables. This happens so infrequently that we probably
can to this manually. NB! Don't use the same locality name if the
locality has changed. For example ``Blomsteringen'' is one unique
location now. If blomsteringen uses different sample locations, we
should split them up with separate names.

This is one row in the container table.

\begin{Shaded}
\begin{Highlighting}[]
\NormalTok{containers <-}\StringTok{ }\KeywordTok{tbl}\NormalTok{(con, }\KeywordTok{in_schema}\NormalTok{(}\StringTok{"common"}\NormalTok{, }\StringTok{"containers"}\NormalTok{))}

\NormalTok{containers }\OperatorTok\StringTok{ }
\StringTok{  }\KeywordTok{print}\NormalTok{(}\DataTypeTok{n =} \DecValTok{1}\NormalTok{,}
        \DataTypeTok{width =} \OtherTok{Inf}\NormalTok{)}
\end{Highlighting}
\end{Shaded}

\begin{verbatim}
## # Source:   table<common.containers> [?? x 27]
## # Database: postgres [jens.astrom@ninradardata01.nina.no:5432/planteimport]
##   id                                   container subsample locality          
##   <chr>                                    <int>     <int> <chr>             
## 1 a88aa624-2200-11e8-8b3b-001cc4ddf696        29         5 Plantasjen Skedsmo
##   date_sampled date_in    date_out   netting_type species_latin      
##   <date>       <date>     <date>     <chr>        <chr>              
## 1 2015-04-26   2015-04-27 2015-04-30 F            Cardamine pratensis
##   plant_comment wet_volume wet_weight dry_volume dry_weight exporter        
##   <chr>              <dbl>      <dbl>      <dbl>      <dbl> <chr>           
## 1 viftelønn              2        554        1.5        198 Plantagen source
##   country transport_type pdf_present mattilsynet comment_certificate oh_recieved
##   <chr>   <chr>          <lgl>       <lgl>       <chr>               <chr>      
## 1 Germany <NA>           TRUE        FALSE       <NA>                <NA>       
##   container_weight number_of_articles number_of_species number_total
##              <dbl>              <int>             <int>        <int>
## 1               NA                  7                82         3517
##   volume_per_crate comment_contents                
##   <chr>            <chr>                           
## 1 <NA>             Litt usikker på sertifikatet her
## # ... with more rows
\end{verbatim}

When punching, you don't need to fill the ``id'' column. That is
internal to the database. I'll write out a small sample that could be
used for punching (but the one we already use is fine!).

\begin{Shaded}
\begin{Highlighting}[]
\NormalTok{containers }\OperatorTok\StringTok{ }
\StringTok{  }\KeywordTok{select}\NormalTok{(}\OperatorTok{-}\NormalTok{id) }\OperatorTok\StringTok{  }\CommentTok{#internal primary key in database}
\StringTok{  }\KeywordTok{collect}\NormalTok{() }\OperatorTok\StringTok{ }
\StringTok{  }\KeywordTok{slice}\NormalTok{(}\DecValTok{1}\OperatorTok{:}\DecValTok{5}\NormalTok{) }\OperatorTok\StringTok{ }
\StringTok{  }\KeywordTok{write_csv}\NormalTok{(}\DataTypeTok{path =} \StringTok{"out/example_containers.csv"}\NormalTok{)}
\end{Highlighting}
\end{Shaded}

\hypertarget{insect-container-records}{%
\section{Insect container records}\label{insect-container-records}}

This table holds all invertebrate data from the containers. So far, we
have only imported the container records in the database, and we will
make similar tables for the other collected invertebrates.

\begin{Shaded}
\begin{Highlighting}[]
\NormalTok{insect_container_records <-}\StringTok{ }\KeywordTok{tbl}\NormalTok{(con, }\KeywordTok{in_schema}\NormalTok{(}\StringTok{"insects"}\NormalTok{, }\StringTok{"container_records"}\NormalTok{))}
\end{Highlighting}
\end{Shaded}

There are some colums that are filled automatically in the database; id,
latinsknavnid, last\_updated\_by, last\_updated. Project\_id is left
blank for now, but could be specified if we want to separate the
findings from different projects (contracts?). So these columns are the
only ones that we need to punch (project id can be left blank):

\begin{Shaded}
\begin{Highlighting}[]
\NormalTok{insect_container_records }\OperatorTok\StringTok{ }
\StringTok{  }\KeywordTok{select}\NormalTok{(}\OperatorTok{-}\KeywordTok{c}\NormalTok{(id, latinsknavnid, last_updated_by, last_updated)) }\OperatorTok\StringTok{ }
\StringTok{  }\KeywordTok{arrange}\NormalTok{(container, subsample, species_latin) }
\end{Highlighting}
\end{Shaded}

\begin{verbatim}
## # Source:     lazy query [?? x 5]
## # Database:   postgres [jens.astrom@ninradardata01.nina.no:5432/planteimport]
## # Ordered by: container, subsample, species_latin
##    container subsample projectid species_latin         amount
##        <int>     <int> <chr>     <chr>                  <int>
##  1         1         1 <NA>      Amischa analis             1
##  2         1         1 <NA>      Bourletiella sp. juv.      1
##  3         1         1 <NA>      Desoria grisea             1
##  4         1         1 <NA>      Isotomurus palustris      10
##  5         1         1 <NA>      Proisotoma minuta          1
##  6         1         1 <NA>      Tomocerus vulgaris         1
##  7         1         4 <NA>      Bourletiella sp. juv.      3
##  8         1         4 <NA>      Cartodere bifasciata       1
##  9         1         4 <NA>      Isotomurus palustris      18
## 10         1         4 <NA>      Isotomurus sp. juv.        6
## # ... with more rows
\end{verbatim}

I'll write out a short sample that could be used for future punching.

\begin{Shaded}
\begin{Highlighting}[]
\NormalTok{insect_container_records }\OperatorTok\StringTok{ }
\StringTok{  }\KeywordTok{select}\NormalTok{(}\OperatorTok{-}\KeywordTok{c}\NormalTok{(id, latinsknavnid, last_updated_by, last_updated)) }\OperatorTok\StringTok{ }
\StringTok{  }\KeywordTok{arrange}\NormalTok{(container, subsample, species_latin) }\OperatorTok\StringTok{ }
\StringTok{  }\KeywordTok{collect}\NormalTok{() }\OperatorTok\StringTok{ }
\StringTok{  }\KeywordTok{slice}\NormalTok{(}\DecValTok{1}\OperatorTok{:}\DecValTok{5}\NormalTok{) }\OperatorTok\StringTok{ }
\StringTok{  }\KeywordTok{write_csv}\NormalTok{(}\DataTypeTok{path =} \StringTok{"out/example_insect_container_records.csv"}\NormalTok{)}
\end{Highlighting}
\end{Shaded}

\hypertarget{insect-species-names}{%
\section{Insect species names}\label{insect-species-names}}

All invertebrate records need to conform to a list of species names,
stored in another table. As new species are discovered, we will add
these to the list of names. This list of names is compared to
artsnavneliste from artsdatabanken. In the future we will add the black
list to the database to be able to update the black list categories
automatically (as automatic as possible). Other alien species lists are
also possible to import. We have both a column called ``native'' and one
called ``alien''. They are not 100\% exclusive, as it appears that some
species may not be found earlier in Norway, but still isn't known to be
alien.

Note that we use a single column called \textbf{species\_latin} for the
unique names of the species or ``taxa'' we find. This is the columns
that are used to match species between tables. In case of juveniles, we
specify the juvenile status at the end of the species names, as can be
seen in the example. Yes, there still is some cleaning up to do here.

There are quite a few colums in this table that is either filled
automatically, or could be filled once and for all by me, using matches
from the artsnavnebase. For new species, we will add new rows to this
table. Currently, it would be good to fill out all these lines:

\begin{Shaded}
\begin{Highlighting}[]
\NormalTok{insect_species <-}\StringTok{ }\KeywordTok{tbl}\NormalTok{(con, }\KeywordTok{in_schema}\NormalTok{(}\StringTok{"insects"}\NormalTok{, }\StringTok{"species"}\NormalTok{))}
\end{Highlighting}
\end{Shaded}

\begin{Shaded}
\begin{Highlighting}[]
\NormalTok{insect_species }\OperatorTok\StringTok{ }
\StringTok{  }\KeywordTok{collect}\NormalTok{() }\OperatorTok\StringTok{ }
\StringTok{  }\KeywordTok{group_by}\NormalTok{(stadium) }\OperatorTok\StringTok{ }
\StringTok{  }\KeywordTok{slice}\NormalTok{(}\DecValTok{1}\NormalTok{) }\OperatorTok\StringTok{ }
\StringTok{  }\KeywordTok{select}\NormalTok{(species_latin,}
\NormalTok{         stadium,}
\NormalTok{         indetermined,}
\NormalTok{         autorstring,}
\NormalTok{         native,}
\NormalTok{         alien,}
\NormalTok{         blacklist_cat)}
\end{Highlighting}
\end{Shaded}

\begin{verbatim}
## # A tibble: 6 x 7
## # Groups:   stadium [6]
##   species_latin     stadium  indetermined autorstring native alien blacklist_cat
##   <chr>             <chr>    <lgl>        <chr>       <lgl>  <lgl> <chr>        
## 1 Supraphorura fur~ adult    FALSE        <NA>        FALSE  FALSE <NA>         
## 2 Thysanoptera spp~ juvenile TRUE         <NA>        FALSE  FALSE <NA>         
## 3 Elateridae spp. ~ larvae   TRUE         <NA>        FALSE  FALSE <NA>         
## 4 Diptera spp. juv. larvae ~ TRUE         <NA>        FALSE  FALSE <NA>         
## 5 Sternorrhyncha s~ nymph    TRUE         <NA>        FALSE  FALSE <NA>         
## 6 Aphiodoidea spp.~ nymph a~ TRUE         <NA>        FALSE  FALSE <NA>
\end{verbatim}

I here include the complete list of the ``species names'' known in the
database as of today. \textbf{All new records should use these names} if
there isn't a true new ``species''.

\begin{Shaded}
\begin{Highlighting}[]
\NormalTok{insect_species }\OperatorTok\StringTok{ }
\StringTok{  }\KeywordTok{arrange}\NormalTok{(species_latin) }\OperatorTok\StringTok{ }
\StringTok{  }\KeywordTok{select}\NormalTok{(species_latin,}
\NormalTok{         stadium,}
\NormalTok{         indetermined,}
\NormalTok{         autorstring,}
\NormalTok{         native,}
\NormalTok{         alien,}
\NormalTok{         blacklist_cat) }\OperatorTok\StringTok{ }
\StringTok{  }\KeywordTok{collect}\NormalTok{() }\OperatorTok\StringTok{ }
\StringTok{  }\KeywordTok{write_csv}\NormalTok{(}\DataTypeTok{path =} \StringTok{"out/example_insect_species_names.csv"}\NormalTok{)}
\end{Highlighting}
\end{Shaded}

\hypertarget{plant-records}{%
\section{Plant records}\label{plant-records}}

The plant records are a little bit simpler, but here we need to make a
note of if they where found before or after vernalisation.

\begin{Shaded}
\begin{Highlighting}[]
\NormalTok{plant_container_records <-}\StringTok{ }\KeywordTok{tbl}\NormalTok{(con, }\KeywordTok{in_schema}\NormalTok{(}\StringTok{"plants"}\NormalTok{, }\StringTok{"container_records"}\NormalTok{))}

\NormalTok{plant_container_records }\OperatorTok\StringTok{ }
\StringTok{  }\KeywordTok{select}\NormalTok{(}\OperatorTok{-}\KeywordTok{c}\NormalTok{(id, }
\NormalTok{            last_updated_by,}
\NormalTok{            last_updated))}
\end{Highlighting}
\end{Shaded}

\begin{verbatim}
## # Source:   lazy query [?? x 7]
## # Database: postgres [jens.astrom@ninradardata01.nina.no:5432/planteimport]
##    container subsample latinsknavnid projectid species_latin amount
##        <int>     <int> <int64>       <chr>     <chr>          <int>
##  1         1         1 NA            <NA>      Cardamine hi~      4
##  2         1         1 NA            <NA>      Cardamine hi~     79
##  3         1         1 NA            <NA>      Cerastium gl~      3
##  4         1         1 NA            <NA>      Cerastium gl~      4
##  5         1         1 NA            <NA>      Senecio vulg~     22
##  6         1         1 NA            <NA>      Stellaria me~      1
##  7         1         1 NA            <NA>      Stellaria me~     17
##  8         1         2 NA            <NA>      Cardamine hi~      1
##  9         1         2 NA            <NA>      Juncus bulbo~      1
## 10         1         3 NA            <NA>      Cardamine hi~      1
## # ... with more rows, and 1 more variable: vernalisation <lgl>
\end{verbatim}

I'll write out a short sample that could be used for future punching.

\begin{Shaded}
\begin{Highlighting}[]
\NormalTok{plant_container_records }\OperatorTok\StringTok{ }
\StringTok{  }\KeywordTok{select}\NormalTok{(}\OperatorTok{-}\KeywordTok{c}\NormalTok{(id, }
\NormalTok{            latinsknavnid,}
\NormalTok{            projectid,}
\NormalTok{            last_updated_by,}
\NormalTok{            last_updated)) }\OperatorTok\StringTok{ }
\StringTok{  }\KeywordTok{collect}\NormalTok{() }\OperatorTok\StringTok{ }
\StringTok{  }\KeywordTok{slice}\NormalTok{(}\DecValTok{1}\OperatorTok{:}\DecValTok{5}\NormalTok{) }\OperatorTok\StringTok{ }
\StringTok{  }\KeywordTok{write_csv}\NormalTok{(}\DataTypeTok{path =} \StringTok{"out/example_plant_container_records.csv"}\NormalTok{)}
\end{Highlighting}
\end{Shaded}

\hypertarget{plant-species-names}{%
\section{Plant species names}\label{plant-species-names}}

The plant species names are stored in a simpler table than the insects.
That's because the original insect table was more complex to start with.

\begin{Shaded}
\begin{Highlighting}[]
\NormalTok{plant_species <-}\StringTok{ }\KeywordTok{tbl}\NormalTok{(con, }\KeywordTok{in_schema}\NormalTok{(}\StringTok{"plants"}\NormalTok{, }\StringTok{"species"}\NormalTok{)) }
\end{Highlighting}
\end{Shaded}

For new plant species, it would be good to record at least the columns
listed below. NB, autorstring is not important for me, but could be
added if you like. Like for insects, the separate columns for native and
alien is useful when the status is ``complicated'' or we haven't
identified the specimen to species. I will add a column ``indetermined''
here as well (need to get datahjelp to change the ownership of the
table).

\begin{Shaded}
\begin{Highlighting}[]
\NormalTok{plant_species }\OperatorTok\StringTok{ }
\StringTok{  }\KeywordTok{select}\NormalTok{(species_latin,}
\NormalTok{         species_norsk,}
\NormalTok{         autorstreng,}
\NormalTok{         native,}
\NormalTok{         alien)}
\end{Highlighting}
\end{Shaded}

\begin{verbatim}
## # Source:   lazy query [?? x 5]
## # Database: postgres [jens.astrom@ninradardata01.nina.no:5432/planteimport]
##    species_latin           species_norsk            autorstreng native alien
##    <chr>                   <chr>                    <chr>       <lgl>  <lgl>
##  1 Panicum capillare       Buskkinapalme/dvergpalme <NA>        FALSE  TRUE 
##  2 Erica cf. ciliaris      duskamarant              <NA>        FALSE  TRUE 
##  3 Humulus cf. yunnanensis sommerfuglbusk           <NA>        FALSE  TRUE 
##  4 Panicum capillaceum     Dvergsypress             <NA>        FALSE  TRUE 
##  5 Panicum repens          Dvergpalme               <NA>        FALSE  TRUE 
##  6 Howea sp.               <NA>                     <NA>        FALSE  TRUE 
##  7 Chaenorhinum minus      småtorskemunn            <NA>        FALSE  TRUE 
##  8 Chenopodiastrum murale  gatemelde                <NA>        FALSE  TRUE 
##  9 Chenopodium ficifolium  fikenmelde               <NA>        FALSE  TRUE 
## 10 Chenopodium hircinum    bukkemelde               <NA>        FALSE  TRUE 
## # ... with more rows
\end{verbatim}

I'll write out a complete list of all plant species known in the
database so far. Also here, there are some cleaning up to do.

\begin{Shaded}
\begin{Highlighting}[]
\NormalTok{plant_species }\OperatorTok\StringTok{ }
\StringTok{  }\KeywordTok{arrange}\NormalTok{(species_latin) }\OperatorTok\StringTok{ }
\StringTok{    }\KeywordTok{select}\NormalTok{(species_latin,}
\NormalTok{           species_norsk,}
\NormalTok{           autorstreng,}
\NormalTok{           native,}
\NormalTok{           alien) }\OperatorTok\StringTok{ }
\StringTok{  }\KeywordTok{collect}\NormalTok{() }\OperatorTok\StringTok{ }
\StringTok{  }\KeywordTok{write_csv}\NormalTok{(}\DataTypeTok{path =} \StringTok{"out/example_plant_species_names.csv"}\NormalTok{)}
\end{Highlighting}
\end{Shaded}

Happy punching!


\end{document}
